\documentclass[aspectratio=169, t]{beamer}  % [t], [c], или [b] --- вертикальное выравнивание на слайдах (верх, центр, низ)
%\documentclass[handout]{beamer} % Раздаточный материал (на слайдах всё сразу)

% Настройки + тема оформления находятся в файле preamble.tex
\include{preamble.tex}

% Информация для титульного слайда находится в файле title.tex
\title{DNA gyrase activity}
\subtitle{Single-molecule imaging of DNA gyrase activity \\ in living \textit{Escherichia coli}}
\author{Putin V.V.}
\date{\today}
\institute[MSU Faculty of Biology]
            


\begin{document}

% Титульный слайд
\frame[plain]{\titlepage}

% Содержание презентации
\begin{frame}
    \frametitle{Содержание}
    \begin{multicols}{2}
    \tableofcontents
    \end{multicols}
    % Или вместо автоматического содержания можно создать список вручную:
    %\begin{itemize}
    %    \item Первый раздел
    %    \item Второй раздел
    %    \item Третий раздел
    %\end{itemize}
\end{frame}

% Раздел 1
\section*{Введение} % Не будет учитываться в tableofcontent
\subsection{Подраздел введения}

\begin{frame}
    \frametitle{\insertsection}
    \framesubtitle{\insertsubsection}
    
    % Пример простого слайда с маркированным списком
    \begin{itemize}
        \item Первый пункт
        \item Второй пункт
        \item Третий пункт с \textbf{выделением}
    \end{itemize}
    
\end{frame}

% Раздел 2
\section{Слайд с колонками}
\subsection{Текст и изображение}

\begin{frame}
    \frametitle{\insertsection}
    \framesubtitle{\insertsubsection}
    
    \begin{columns}
        \column{0.6\textwidth}
        % Левая колонка с текстом
        \begin{itemize}
            \item Первый пункт
            \item Второй пункт
            \item Третий пункт
        \end{itemize}
        
        \column{0.4\textwidth}
        % Правая колонка с изображением
        \begin{figure}
            \centering
            \includegraphics[width=0.9\textwidth]{images/logo.png}
            \caption{Подпись к изображению}
        \end{figure}
    \end{columns}
    
\end{frame}

% Раздел 3
\section{Математические формулы}

\begin{frame}
    \frametitle{\insertsection}
    
    % Пример встроенной формулы
    Формула в тексте: $E = mc^2$
    
    % Пример выделенной формулы
    \begin{equation}
        F = G \frac{m_1 m_2}{r^2}
    \end{equation}
    
    % Пример системы уравнений
    \begin{align}
        x^2 + y^2 &= 1 \\
        x + y &= 1
    \end{align}
    
\end{frame}

% Раздел 4
\section{Таблицы и блоки}

\begin{frame}
    \frametitle{\insertsection}
    
    % Пример таблицы
    \begin{table}
        \centering
        \begin{tabular}{|l|c|r|}
            \hline
            \textbf{Левый} & \textbf{Центр} & \textbf{Правый} \\
            \hline
            A1 & B1 & C1 \\
            A2 & B2 & C2 \\
            \hline
        \end{tabular}
        \caption{Пример таблицы}
    \end{table}
    
    % Пример блока (box)
    \begin{block}{Важный блок}
        Содержимое блока с важной информацией.
    \end{block}
    
    \begin{alertblock}{Внимание!}
        Блок для привлечения особого внимания.
    \end{alertblock}
    
\end{frame}

% Заключительный слайд
\section*{Заключение} % Не будет учитываться в tableofcontent

\begin{frame}
    \frametitle{Спасибо за внимание!}
    \begin{center}
        \huge{Вопросы?}
        
        \vspace{1cm}
        \normalsize{\texttt{email@example.com}}
    \end{center}
\end{frame}

\end{document}
